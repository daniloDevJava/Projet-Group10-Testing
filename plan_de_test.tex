\documentclass[a4paper,12pt]{article}
\usepackage[utf8]{inputenc}
\usepackage[T1]{fontenc}
\usepackage{longtable}
\usepackage{geometry}
\geometry{margin=2.5cm}
\usepackage{array}
\usepackage{titlesec}
\usepackage{xcolor}

\titleformat{\section}{\normalfont\Large\bfseries}{\thesection}{1em}{}

\title{\textbf{Plan de Test - Application Web}}
\author{}
\date{}

\begin{document}

\maketitle

\section*{1. Introduction}
Ce document décrit le plan de test pour une application web composée d'une partie frontend (testée avec Playwright) et d'une partie backend (API REST testée avec JUnit). L'objectif est d'assurer la qualité fonctionnelle et technique de l'application à travers des cas de tests bien définis.

\section*{2. Objectifs des tests}
\begin{itemize}
    \item Vérifier le bon fonctionnement des fonctionnalités critiques du frontend (authentification, gestion des véhicules, gestion des utilisateurs).
    \item Garantir la robustesse des endpoints du backend.
    \item Détecter les erreurs d'entrée et les mauvaises pratiques utilisateur.
\end{itemize}

\section*{3. Approche de test}

L’approche de test adoptée repose sur trois niveaux complémentaires :

\begin{itemize}
    \item \textbf{Tests unitaires (Backend)} : réalisés avec \textbf{JUnit 5}. Ces tests portent sur des composants isolés (méthodes de contrôleurs et services). Les dépendances sont simulées à l’aide de \textbf{doublures de test} (\texttt{@MockBean} avec \textbf{Mockito}).
    
    \begin{itemize}
        \item La suite de tests \texttt{UserControllerTest} \textbf{par daniloDevJava} est une suite de tests unitaires qui utilise une doublure du composant \texttt{UserService}.
        \item La suite de tests \texttt{VehiculeControllerTest} \textbf{par chenjoProsper}est une suite de tests unitaires qui utilise une doublure du composant \texttt{VehiculeService}.
    \end{itemize}
    
    \item \textbf{Tests d'intégration (Backend)} : réalisés avec \textbf{Spring Boot Test} et une base de données \textbf{H2} en mémoire. Ces tests permettent de vérifier la collaboration entre plusieurs couches de l'application (service, repository).

    \item \textbf{Tests end-to-end (Frontend)} : automatisés avec \textbf{Playwright}. Ces tests simulent des scénarios utilisateurs réels dans un navigateur (par exemple : connexion, affichage des véhicules, ajout d’un utilisateur).

\end{itemize}

Tous les tests sont automatisés et peuvent être intégrés dans le pipeline d’intégration continue (CI).


\section*{4. Environnement de test}
\begin{itemize}
    \item \textbf{Frontend} : Playwright, navigateur Chrome/Chromium.
    \item \textbf{Backend} : JUnit 5, Mockito, Spring Boot.
    \item Base de données H2 (en mémoire pour les tests).
\end{itemize}

\section*{5. Données de test}
Les jeux de données incluent :
\begin{itemize}
    \item Utilisateurs avec différents niveaux de validité (email correct/incorrect, mot de passe fort/faible).
    \item Véhicules avec des prix, immatriculations et modèles variés.
\end{itemize}

\section*{6. Cas de test Frontend (Playwright)}
\begin{longtable}{|p{0.1\textwidth}|p{0.8\textwidth}|}
\hline
\textbf{ID} & \textbf{Description du test} \\
\hline
F1 & Connexion : vérifier qu’un utilisateur peut se connecter avec des identifiants valides. \\
\hline
F2 & Véhicules : vérifier que la liste des véhicules s’affiche après authentification. \\
\hline
F3 & Utilisateurs : vérifier que l’ajout d’un utilisateur fonctionne et que celui-ci apparaît dans la liste. \\
\hline
\end{longtable}

\section*{7. Cas de test Backend - \texttt{UserControllerTests}}
\begin{longtable}{|p{0.1\textwidth}|p{0.8\textwidth}|}
\hline
\textbf{ID} & \textbf{Description du test} \\
\hline
U1 & Succès \texttt{/users/add} : statut 201, JSON contenant \texttt{id}, \texttt{email}, \texttt{name (en majuscules)}. \\
\hline
U2 & Échec \texttt{/users/add} : email mal formé $\Rightarrow$ 400, JSON erreur. \\
\hline
U3 & Échec \texttt{/users/add} : mot de passe faible $\Rightarrow$ 400, JSON erreur. \\
\hline
U4 & Succès \texttt{/login} : statut 200, JSON avec \texttt{accessToken} et \texttt{refreshToken}. \\
\hline
U5 & Échec \texttt{/login} : identifiants invalides $\Rightarrow$ 404. \\
\hline
U6 & Succès \texttt{/refresh-access-tokens} : token valide $\Rightarrow$ 200, réponse $>$ 50 caractères. \\
\hline
U7 & Échec token invalide $\Rightarrow$ 403, message : \textit{Le refresh token est invalide}. \\
\hline
U8 & Échec token expiré $\Rightarrow$ 403, message : \textit{Le refresh token est expiré}. \\
\hline
U9 & Succès \texttt{/refresh-tokens} $\Rightarrow$ 200, JSON avec nouveaux tokens. \\
\hline
U10 & Échec \texttt{/refresh-tokens} avec token expiré $\Rightarrow$ 403, JSON erreur. \\
\hline
U11 & Échec \texttt{/refresh-tokens} avec token invalide $\Rightarrow$ 403, JSON erreur. \\
\hline
U12 & Succès \texttt{/users/\{id\}} (PUT) : mise à jour, statut 200, JSON mis à jour. \\
\hline
U13 & Échec mise à jour : email invalide $\Rightarrow$ 400, JSON erreur. \\
\hline
U14 & Échec mise à jour : utilisateur inexistant $\Rightarrow$ 404. \\
\hline
U15 & Succès \texttt{/users/all} : liste utilisateurs, statut 200, tableau JSON. \\
\hline
U16 & Succès \texttt{/change-password} $\Rightarrow$ 200, message : ``mot de passe modifié''. \\
\hline
U17 & Échec \texttt{/change-password} : nouveau mot de passe faible $\Rightarrow$ 400, JSON erreur. \\
\hline
U18 & Échec \texttt{/change-password} : ancien mot de passe incorrect $\Rightarrow$ 400, JSON erreur. \\
\hline
\end{longtable}






\section*{8. Cas de test Backend - \texttt{VehiculeControllerTests} }
\begin{longtable}{|p{0.1\textwidth}|p{0.8\textwidth}|}
\hline
\textbf{ID} & \textbf{Description du test} \\
\hline
V1 & Succès \texttt{/vehicules/create} : statut 201, JSON avec données du véhicule. \\
\hline
V2 & Échec création : champs vides ou prix négatif $\Rightarrow$ 400, tableau d'objets JSON d'erreurs. \\
\hline
V3 & Échec création doublon registerNumber : premier succès, second rejeté avec erreur structurée. \\
\hline
V4 & Succès \texttt{/vehicule/\{id\}} (GET) : statut 200, JSON véhicule. \\
\hline
V5 & Succès suppression \texttt{/vehicule/\{id\}} : statut 200, message : ``vehicule is deleted successfully''. \\
\hline
V6 & Échec suppression : id inconnu $\Rightarrow$ 404. \\
\hline
V7 & Échec récupération par ID inconnu $\Rightarrow$ 404. \\
\hline
V8 & Succès \texttt{/vehicule/search-by-price} : statut 200, tableau JSON filtré. \\
\hline
V9 & Succès \texttt{/vehicule/number/\{registerNum\}} : statut 200, JSON véhicule. \\
\hline
V10 & Échec \texttt{/vehicule/number/\{registerNum\}} : registre inconnu $\Rightarrow$ 404. \\
\hline
V11 & Succès update \texttt{/vehicule/\{id\}} : statut 200, JSON véhicule mis à jour. \\
\hline
\end{longtable}

\section{9. Cas de test Backend - \texttt{UserServiceTest} }
\begin{longtable}{|p{0.1\textwidth}|p{0.8\textwidth}|}
\hline
\textbf{ID} &  \textbf{statut} & \textbf{MUT} & \textbf{Description du test} & \textbf{assertion(s)}  \\
\hline
Us1 & Succès & \textbf{userService.createUser(user)} & \texttt{Test d'ajout succès d'user dans la bd} & \texttt{L'objet created doit etre non nul} \\
\hline
Us2 & Succès & \textbf{userService.updateUser(userDto,id)} & \texttt{Test de mise a jour reussie d'un utilisateur} & 
\hline
\end{longtable}
\section*{10. Risques et problèmes potentiels}
\begin{itemize}
    \item Mauvaise gestion des tokens expirés.
    \item Incohérences entre les formats JSON du frontend et du backend.
    \item Erreurs silencieuses non détectées sans journalisation explicite.
\end{itemize}
 
\section*{11. Reporting et communication}
\begin{itemize}
    \item Les résultats des tests seront analysés et communiqués après chaque exécution.
    \item Formats de rapport : HTML et JSON.
    \item Outils utilisés : CI/CD avec GitHub Actions.
\end{itemize}

\section*{12. Conclusion}
Ce plan de test couvre les scénarios principaux de l'application. Il est destiné à être enrichi progressivement, au fur et à mesure de l'ajout de nouvelles fonctionnalités ou de la détection de bugs.

\end{document}

